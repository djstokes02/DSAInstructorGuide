% Options for packages loaded elsewhere
\PassOptionsToPackage{unicode}{hyperref}
\PassOptionsToPackage{hyphens}{url}
\PassOptionsToPackage{dvipsnames,svgnames,x11names}{xcolor}
%
\documentclass[
  letterpaper,
  DIV=11,
  numbers=noendperiod]{scrreprt}

\usepackage{amsmath,amssymb}
\usepackage{iftex}
\ifPDFTeX
  \usepackage[T1]{fontenc}
  \usepackage[utf8]{inputenc}
  \usepackage{textcomp} % provide euro and other symbols
\else % if luatex or xetex
  \usepackage{unicode-math}
  \defaultfontfeatures{Scale=MatchLowercase}
  \defaultfontfeatures[\rmfamily]{Ligatures=TeX,Scale=1}
\fi
\usepackage{lmodern}
\ifPDFTeX\else  
    % xetex/luatex font selection
\fi
% Use upquote if available, for straight quotes in verbatim environments
\IfFileExists{upquote.sty}{\usepackage{upquote}}{}
\IfFileExists{microtype.sty}{% use microtype if available
  \usepackage[]{microtype}
  \UseMicrotypeSet[protrusion]{basicmath} % disable protrusion for tt fonts
}{}
\makeatletter
\@ifundefined{KOMAClassName}{% if non-KOMA class
  \IfFileExists{parskip.sty}{%
    \usepackage{parskip}
  }{% else
    \setlength{\parindent}{0pt}
    \setlength{\parskip}{6pt plus 2pt minus 1pt}}
}{% if KOMA class
  \KOMAoptions{parskip=half}}
\makeatother
\usepackage{xcolor}
\setlength{\emergencystretch}{3em} % prevent overfull lines
\setcounter{secnumdepth}{-\maxdimen} % remove section numbering
% Make \paragraph and \subparagraph free-standing
\makeatletter
\ifx\paragraph\undefined\else
  \let\oldparagraph\paragraph
  \renewcommand{\paragraph}{
    \@ifstar
      \xxxParagraphStar
      \xxxParagraphNoStar
  }
  \newcommand{\xxxParagraphStar}[1]{\oldparagraph*{#1}\mbox{}}
  \newcommand{\xxxParagraphNoStar}[1]{\oldparagraph{#1}\mbox{}}
\fi
\ifx\subparagraph\undefined\else
  \let\oldsubparagraph\subparagraph
  \renewcommand{\subparagraph}{
    \@ifstar
      \xxxSubParagraphStar
      \xxxSubParagraphNoStar
  }
  \newcommand{\xxxSubParagraphStar}[1]{\oldsubparagraph*{#1}\mbox{}}
  \newcommand{\xxxSubParagraphNoStar}[1]{\oldsubparagraph{#1}\mbox{}}
\fi
\makeatother


\providecommand{\tightlist}{%
  \setlength{\itemsep}{0pt}\setlength{\parskip}{0pt}}\usepackage{longtable,booktabs,array}
\usepackage{calc} % for calculating minipage widths
% Correct order of tables after \paragraph or \subparagraph
\usepackage{etoolbox}
\makeatletter
\patchcmd\longtable{\par}{\if@noskipsec\mbox{}\fi\par}{}{}
\makeatother
% Allow footnotes in longtable head/foot
\IfFileExists{footnotehyper.sty}{\usepackage{footnotehyper}}{\usepackage{footnote}}
\makesavenoteenv{longtable}
\usepackage{graphicx}
\makeatletter
\def\maxwidth{\ifdim\Gin@nat@width>\linewidth\linewidth\else\Gin@nat@width\fi}
\def\maxheight{\ifdim\Gin@nat@height>\textheight\textheight\else\Gin@nat@height\fi}
\makeatother
% Scale images if necessary, so that they will not overflow the page
% margins by default, and it is still possible to overwrite the defaults
% using explicit options in \includegraphics[width, height, ...]{}
\setkeys{Gin}{width=\maxwidth,height=\maxheight,keepaspectratio}
% Set default figure placement to htbp
\makeatletter
\def\fps@figure{htbp}
\makeatother

\KOMAoption{captions}{tableheading}
\makeatletter
\@ifpackageloaded{caption}{}{\usepackage{caption}}
\AtBeginDocument{%
\ifdefined\contentsname
  \renewcommand*\contentsname{Table of contents}
\else
  \newcommand\contentsname{Table of contents}
\fi
\ifdefined\listfigurename
  \renewcommand*\listfigurename{List of Figures}
\else
  \newcommand\listfigurename{List of Figures}
\fi
\ifdefined\listtablename
  \renewcommand*\listtablename{List of Tables}
\else
  \newcommand\listtablename{List of Tables}
\fi
\ifdefined\figurename
  \renewcommand*\figurename{Figure}
\else
  \newcommand\figurename{Figure}
\fi
\ifdefined\tablename
  \renewcommand*\tablename{Table}
\else
  \newcommand\tablename{Table}
\fi
}
\@ifpackageloaded{float}{}{\usepackage{float}}
\floatstyle{ruled}
\@ifundefined{c@chapter}{\newfloat{codelisting}{h}{lop}}{\newfloat{codelisting}{h}{lop}[chapter]}
\floatname{codelisting}{Listing}
\newcommand*\listoflistings{\listof{codelisting}{List of Listings}}
\makeatother
\makeatletter
\makeatother
\makeatletter
\@ifpackageloaded{caption}{}{\usepackage{caption}}
\@ifpackageloaded{subcaption}{}{\usepackage{subcaption}}
\makeatother

\ifLuaTeX
  \usepackage{selnolig}  % disable illegal ligatures
\fi
\usepackage{bookmark}

\IfFileExists{xurl.sty}{\usepackage{xurl}}{} % add URL line breaks if available
\urlstyle{same} % disable monospaced font for URLs
\hypersetup{
  colorlinks=true,
  linkcolor={blue},
  filecolor={Maroon},
  citecolor={Blue},
  urlcolor={Blue},
  pdfcreator={LaTeX via pandoc}}


\author{}
\date{}

\begin{document}

<!-- Formating image above the title and adding the DSA website link -->

<div style="text-align: center; margin-bottom: 20px;">
   <a href="https://datascienceacademy.ncsu.edu/courses/upcoming-dsa-courses/" target="_blank">
       <img src="DSA-lockUp-2024.png" alt="Header Image" style="max-width: 525px; height: auto;">
   </a>
</div>


\chapter{The Guide}\label{the-guide}

This resource is designed as a guide for instructors in the Data Science
and AI Academy at NC State. In this guide you will find information
about DSA course expectations, course structure and implementation
suggestions, teaching and learning resources, university policies and
more.

You can use the table of contents to navigate to a section of interest.

\section{About the DSA}\label{about-the-dsa}

Visit the \href{https://datascienceacademy.ncsu.edu/about/}{About}
section on the \href{https://datascienceacademy.ncsu.edu/}{DSA Website}
to learn a bit about the DSA.

\subsection{DSA Staff - Teaching \&
Learning}\label{dsa-staff---teaching-learning}

\begin{itemize}
\item
  Rachel Levy - Executive Director
\item
  David Stokes - Teaching Coordinator
\item
  Sunghwan Byun - Director of Educational Research
\end{itemize}

\href{https://datascienceacademy.ncsu.edu/about/our-team/}{Our Team}

\subsection{Course Collaboration Leaders
Support.}\label{course-collaboration-leaders-support.}

\textbf{What and who are the DSA course collaboration leaders}?

The DSA Course Collaboration Leaders (CCLs) support DSA teaching and
learning by offering office hours to their peers in DSA courses. They
are a valuable resource for students who may have questions, are seeking
guidance, or discussion \& community.

Course Collaboration Leaders (CCLs) are typically undergraduate students
who have experience with DSA courses (e.g., have taken at least one DSA
course).

\begin{itemize}
\tightlist
\item
  CCL office hours are in-person and on Zoom.
\item
  The CCL office hours schedule is made available each semester and
  instructors should provide this information to students in their
  courses with additional reminders on a weekly basis.
\end{itemize}

The current semester CCL schedule will be given to you so that you can
advertise this resource in your courses and on your course pages. Please
notify the Teaching Coordinator if you have students from your courses
that you want to recommend to become a Course Collaboration Leader in a
future semester or year.

Please forward other inquiries from students wanting to contribute to
the DSA to datascienceacademy@ncsu.edu.

\href{https://docs.google.com/document/d/1VPfx_Ib2ToiO3eP1ipACCrsnoFTWcJNVhCXyESGf5PY/edit?usp=sharing}{Additional
information about graduate student TA inquiries}




\end{document}
